
%%%%%%%%%%%%%%%%%%%%%%%%%%%%%%%%%%%%%%%%%%%%%%%%%%%%%%%%%%%%%%%%%%%%%
%% This is a (brief) model paper using the achemso class
%% The document class accepts keyval options, which should include
%% the target journal and optionally the manuscript type.
%%%%%%%%%%%%%%%%%%%%%%%%%%%%%%%%%%%%%%%%%%%%%%%%%%%%%%%%%%%%%%%%%%%%%
\documentclass[journal=jpccck,manuscript=article]{achemso}

%%%%%%%%%%%%%%%%%%%%%%%%%%%%%%%%%%%%%%%%%%%%%%%%%%%%%%%%%%%%%%%%%%%%%
%% Place any additional packages needed here.  Only include packages
%% which are essential, to avoid problems later. Do NOT use any
%% packages which require e-TeX (for example etoolbox): the e-TeX
%% extensions are not currently available on the ACS conversion
%% servers.
%%%%%%%%%%%%%%%%%%%%%%%%%%%%%%%%%%%%%%%%%%%%%%%%%%%%%%%%%%%%%%%%%%%%%
\usepackage[version=3]{mhchem} % Formula subscripts using \ce{}

%%%%%%%%%%%%%%%%%%%%%%%%%%%%%%%%%%%%%%%%%%%%%%%%%%%%%%%%%%%%%%%%%%%%%
%% If issues arise when submitting your manuscript, you may want to
%% un-comment the next line.  This provides information on the
%% version of every file you have used.
%%%%%%%%%%%%%%%%%%%%%%%%%%%%%%%%%%%%%%%%%%%%%%%%%%%%%%%%%%%%%%%%%%%%%
%%\listfiles

%%%%%%%%%%%%%%%%%%%%%%%%%%%%%%%%%%%%%%%%%%%%%%%%%%%%%%%%%%%%%%%%%%%%%
%% Place any additional macros here.  Please use \newcommand* where
%% possible, and avoid layout-changing macros (which are not used
%% when typesetting).
%%%%%%%%%%%%%%%%%%%%%%%%%%%%%%%%%%%%%%%%%%%%%%%%%%%%%%%%%%%%%%%%%%%%%
%\newcommand*\mycommand[1]{\texttt{\emph{#1}}}

%%%%%%%%%%%%%%%%%%%%%%%%%%%%%%%%%%%%%%%%%%%%%%%%%%%%%%%%%%%%%%%%%%%%%
%% Meta-data block
%% ---------------
%% Each author should be given as a separate \author command.
%%
%% Corresponding authors should have an e-mail given after the author
%% name as an \email command. Phone and fax numbers can be given
%% using \phone and \fax, respectively; this information is optional.
%%
%% The affiliation of authors is given after the authors; each
%% \affiliation command applies to all preceding authors not already
%% assigned an affiliation.
%%
%% The affiliation takes an option argument for the short name.  This
%% will typically be something like "University of Somewhere".
%%
%% The \altaffiliation macro should be used for new address, etc.
%% On the other hand, \alsoaffiliation is used on a per author basis
%% when authors are associated with multiple institutions.
%%%%%%%%%%%%%%%%%%%%%%%%%%%%%%%%%%%%%%%%%%%%%%%%%%%%%%%%%%%%%%%%%%%%%
\author{Elke Fasshauer}
\affiliation[UIT]{Tromso, Norway}
\email{elke.fasshauer@uit.no}
\author{Melanie Mucke}
\altaffiliation{Now at: Department of Physics and Astronomy, Uppsala University, Box 516, 75120 Uppsala, Sweden}
\author{Marko F\"orstel}
\altaffiliation{Now at: University of Hawai'i at Manoa, 2545 McCarthy Mall, 96816 HI Honolulu, USA}
\affiliation[IPP]{Max-Planck-Institute for Plasma Physics, Boltzmannstr. 2, 85748 Garching, Germany}
\author{...}
\author{Uwe Hergenhahn}
\affiliation[IPP HGW]{Max-Planck-Institute for Plasma Physics, Wendelsteinstr. 1, 14791 Greifswald, Germany}
\email{uwe.hergenhahn@ipp.mpg.de}
%\phone{+123 (0)123 4445556}
%\fax{+123 (0)123 4445557}
%\alsoaffiliation[Second University]
%{Department of Chemistry, Second University, Nearby Town}

%%%%%%%%%%%%%%%%%%%%%%%%%%%%%%%%%%%%%%%%%%%%%%%%%%%%%%%%%%%%%%%%%%%%%
%% The document title should be given as usual. Some journals require
%% a running title from the author: this should be supplied as an
%% optional argument to \title.
%%%%%%%%%%%%%%%%%%%%%%%%%%%%%%%%%%%%%%%%%%%%%%%%%%%%%%%%%%%%%%%%%%%%%
\title{One more attempt on ArXe}

%%%%%%%%%%%%%%%%%%%%%%%%%%%%%%%%%%%%%%%%%%%%%%%%%%%%%%%%%%%%%%%%%%%%%
%% Some journals require a list of abbreviations or keywords to be
%% supplied. These should be set up here, and will be printed after
%% the title and author information, if needed.
%%%%%%%%%%%%%%%%%%%%%%%%%%%%%%%%%%%%%%%%%%%%%%%%%%%%%%%%%%%%%%%%%%%%%
%\abbreviations{IR,NMR,UV}
%\keywords{American Chemical Society, \LaTeX}

%%%%%%%%%%%%%%%%%%%%%%%%%%%%%%%%%%%%%%%%%%%%%%%%%%%%%%%%%%%%%%%%%%%%%
%% The manuscript does not need to include \maketitle, which is
%% executed automatically.
%%%%%%%%%%%%%%%%%%%%%%%%%%%%%%%%%%%%%%%%%%%%%%%%%%%%%%%%%%%%%%%%%%%%%
\begin{document}

%%%%%%%%%%%%%%%%%%%%%%%%%%%%%%%%%%%%%%%%%%%%%%%%%%%%%%%%%%%%%%%%%%%%%
%% The "tocentry" environment can be used to create an entry for the
%% graphical table of contents. It is given here as some journals
%% require that it is printed as part of the abstract page. It will
%% be automatically moved as appropriate.
%%%%%%%%%%%%%%%%%%%%%%%%%%%%%%%%%%%%%%%%%%%%%%%%%%%%%%%%%%%%%%%%%%%%%
\begin{tocentry}

Some journals require a graphical entry for the Table of Contents.
This should be laid out ``print ready'' so that the sizing of the
text is correct.

Inside the \texttt{tocentry} environment, the font used is Helvetica
8\,pt, as required by \emph{Journal of the American Chemical
Society}.

The surrounding frame is 9\,cm by 3.5\,cm, which is the maximum
permitted for  \emph{Journal of the American Chemical Society}
graphical table of content entries. The box will not resize if the
content is too big: instead it will overflow the edge of the box.

This box and the associated title will always be printed on a
separate page at the end of the document.

\end{tocentry}

%%%%%%%%%%%%%%%%%%%%%%%%%%%%%%%%%%%%%%%%%%%%%%%%%%%%%%%%%%%%%%%%%%%%%
%% The abstract environment will automatically gobble the contents
%% if an abstract is not used by the target journal.
%%%%%%%%%%%%%%%%%%%%%%%%%%%%%%%%%%%%%%%%%%%%%%%%%%%%%%%%%%%%%%%%%%%%%
\begin{abstract}
\end{abstract}

%%%%%%%%%%%%%%%%%%%%%%%%%%%%%%%%%%%%%%%%%%%%%%%%%%%%%%%%%%%%%%%%%%%%%
%% Start the main part of the manuscript here.
%%%%%%%%%%%%%%%%%%%%%%%%%%%%%%%%%%%%%%%%%%%%%%%%%%%%%%%%%%%%%%%%%%%%%
\section{Introduction}
%
Electron spectroscopy can make important contributions to the research on composition and structure of free nanoparticles.\cite{jpcc} Besides by photoionization, free electrons emerging from nanoparticles can also be produced by the relaxation of electronically excited states. One such process, which is of particular importance in weakly bonded systems, is the Interatomic or Intermolecular Coulombic Decay (ICD).\cite{cederbaum} In ICD, an electronic excitation decays by energy transfer to one of its neighbour atoms or molecules, thus releasing a free electron from the latter site. ICD is an important relaxation channel e.g. for inner-valence holes in elements C-Ne, and also for core levels vacancies e.g. in \ce{H2O}, where it competes with Auger decay\cite{slavicek}. 

By definition, ICD is particularly sensitive to the chemical environment of the atom or molecule in which the primary excitation has taken place. We suggest that this property of ICD decay spectra can be used to derive information on a nanoparticle or a solvation system, for example. We report here about studies in which we produced heterogeneous clusters of the noble gases Ar and Xe with different sizes, and compositions ranging from a few Xe dopant atoms in an Ar matrix to clusters containing an equal amount of both species. By the use of electron, electron coincidence spectroscopy and of {\it ab initio} calculations on prototypical systems we show how the radiationless decay spectrum of Ar inner valence (3s) ionized states connects to the structure of the clusters.

Intermolecular Coulombic Decay initially was predicted from theoretical considerations of the energy levels in singly vs. doubly ionized, and doubly vs. triply ionized, hydrogen bonded clusters.\cite{cederbaum} First experimental work some years later used Ne clusters,\cite{marburger,jahnkenedimer} but quickly was followed by demonstrations of ICD in a diverse range of other systems. Experimental and theoretical progress has been reviewed.\cite{hergenhahn_review, averbukh_review, jahnke_review} ICD proceeds by an initial ionization producing an ion in an excited state, followed by a transfer of energy to a neighbouring site, and an electron emission process therefrom. In the final state a positive charge remains at the site of ionization, and another was created at another atom or molecule. Soon after, related autoionization processes were discovered that proceed via a charge transfer instead of an energy transfer. 
%These were termed `Electron Transfer Mediated Decay' (ETMD),\cite{zobeley,mueller,sakai,foerstel} for transitions in which the originally excited site is neutralized by the decay, and 'exchange ICD' for charge transfer contributions to the normal ICD amplitude.\cite{santrarev,jahnkesat}
We find that both energy and charge transfer induced autoionization plays a role in Ar-Xe, and will detail the relevant processes below. The notion of ICD requires that the electronic orbitals of the two sites can be distinguished, which typically is the case in weakly bonded systems, held together by hydrogen bonds or van-der-Waals bonds. In the case of strong (covalent or metallic) bonding, there is no distinction between ICD and Auger decay.\cite{hergenhahn_review}

First experiments on ICD in heterogeneous systems were on Ne-Ar clusters.\cite{barthnear} Those were followed by studies of ICD-like inner-shell decays in aqueous solution.\cite{aziz,pokapanich,pokapanich2011} Pioneering work also showed the potential for studies of the interface between a substrate and an adsorbate by ICD.\cite{grieves} More detailed work on Ne-Ar clusters was recently presented by some of the authors.\cite{fasshauer2014} For this system, a detailed analysis of photoelectron spectra with respect to structural features of the mixed clusters is also available.\cite{lundwall} In Ref. \citenum{fasshauer2014} it was shown that an analysis of the ICD spectra allowed to decide between structural alternatives for which the photoelectron data were indiscriminate.

Rare gas clusters are suitable prototype systems for studies of (e.g.) ICD, as they can easily be produced by supersonic expansion. Mixed rare gas clusters easily form by coexpansion of the two gases through a cooled nozzle. The size of the clusters can be changed by varying the expansion parameters, although for heterogeneous clusters some uncertainty remains with respect to the cluster size, as empirical scaling laws \cite{hagena1981} apply to expansion of pure gases only. Early studies of Ar-Xe clusters mostly focussed on the extreme cases of a single Xe atom embedded in an Ar cluster (e.g. Ref. \citenum{lengenprl}). Different sites of the dopant atom (on top of a surface, integrated in an Ar surface, inside the cluster) were distinguished by three separate bands in the fluorescence yield recorded as a function of excitation wavelength.\cite{lengenprl} Increasing the Xe concentration in the expansion lead to changes in the fluorescence excitation spectrum that were interpreted as formation of \ce{Xe2} and larger Xe complexes inside the clusters.\cite{lengen} Experiments on Xe-rich mixed clusters were also conducted, and concluded that increasing the Xe content in a coexpanding ArXe mixture leads to the formation of Xe-core, Ar-shell systems with a sharp interface.\cite{tchaplyguine} This finding was confirmed by electron diffraction experiments.\cite{Danylchenko} Always, the observed Xe content in the clusters is above the Xe content in the expanding gas mixture, as Xe can be condensed much easier than Ar.

Most recently, on the experimental side the investigations 

<Lindblad, Fasshauer>


%\begin{figure}
%  As well as the standard float types \texttt{table}\\
%  and \texttt{figure}, the class also recognises\\
%  \texttt{scheme}, \texttt{chart} and \texttt{graph}.
%  \caption{An example figure}
%  \label{fgr:example}
%\end{figure}

%\begin{table}
%  \caption{An example table}
%  \label{tbl:example}
%  \begin{tabular}{ll}
%    \hline
%    Header one  & Header two  \\
%    \hline
%    Entry one   & Entry two   \\
%    Entry three & Entry four  \\
%    Entry five  & Entry five  \\
%    Entry seven & Entry eight \\
%    \hline
%  \end{tabular}
%\end{table}

%%%%%%%%%%%%%%%%%%%%%%%%%%%%%%%%%%%%%%%%%%%%%%%%%%%%%%%%%%%%%%%%%%%%%
%% The "Acknowledgement" section can be given in all manuscript
%% classes.  This should be given within the "acknowledgement"
%% environment, which will make the correct section or running title.
%%%%%%%%%%%%%%%%%%%%%%%%%%%%%%%%%%%%%%%%%%%%%%%%%%%%%%%%%%%%%%%%%%%%%
\begin{acknowledgement}
%
We thank HZB for the allocation of synchrotron radiation beamtime, and the Deutsche Forschungsgemeinschaft for funding via the Forschergruppe 1789.
%
\end{acknowledgement}

%%%%%%%%%%%%%%%%%%%%%%%%%%%%%%%%%%%%%%%%%%%%%%%%%%%%%%%%%%%%%%%%%%%%%
%% The same is true for Supporting Information, which should use the
%% suppinfo environment.
%%%%%%%%%%%%%%%%%%%%%%%%%%%%%%%%%%%%%%%%%%%%%%%%%%%%%%%%%%%%%%%%%%%%%
\begin{suppinfo}

This will usually read something like: ``Experimental procedures and
characterization data for all new compounds. The class will
automatically add a sentence pointing to the information on-line:

\end{suppinfo}

%%%%%%%%%%%%%%%%%%%%%%%%%%%%%%%%%%%%%%%%%%%%%%%%%%%%%%%%%%%%%%%%%%%%%
%% The appropriate \bibliography command should be placed here.
%% Notice that the class file automatically sets \bibliographystyle
%% and also names the section correctly.
%%%%%%%%%%%%%%%%%%%%%%%%%%%%%%%%%%%%%%%%%%%%%%%%%%%%%%%%%%%%%%%%%%%%%
\bibliography{arxe}

\end{document}
