\section{Theoretical Approach}
Experimental coincidence spectra are given by the kinetic energies $E_{sec}$
of the secondary electron
and the intensities of the peaks. The latter is proportional to the
probability of the decay as well as the theoretically determinable
decay width $\Gamma=\frac{\hbar}{\tau}$,
which is inverse proportional to the lifetime $\tau$. The manifold of these
energies and decay widths will therefore compose the electron-electron
coincidence spectrum.
In case of noble gas clusters
with given structures three aspects have to be taken into account:
different decay mechanisms, the possibility to decay with multiple
interaction partners and different decay channels within each decay mechanism.
At the current stage of development we need to neglect the nuclear dynamics which
additionally might play an important role.

Decomposing every system into pairs and triples of atoms is a very useful
first order approximation to both the investigation of energies and
decay widths of a larger system. Pairs and triples are combinations of
two and three atoms, respectively.
These atoms do not necessarily need to form bonds between each other or
even be close, but they are characterized according to fixed internal
coordinates. Each pair and triple can be described by its properties
which are in first order of approximation independent of further, eventually
present, atoms. In the following, this approach is going to be called
model of pairs and triples.

In case of the electronic decay processes one is interested in the
energies of the initial $E_{in}$ and the final states $E_{fin}$ of
the corresponding processes in order to determine, whether a channel $\beta$
is open, i.e., in accordance with energy conservation, or not. When the channel
is open, the excess energy is carried away by the emitted electron $E_{sec}$
in form of its kinetic energy. These energies can in the model
of pairs and triples be approximated to be

\begin{align}
 E_{in}^\beta  &= SIP(X_{in}^\beta) \label{equation:E_in}\\
 E_{fin}^\beta &= SIP(X_{fin1}^\beta) + SIP(X_{fin2}^\beta) + \frac 1d
           \label{equation:E_fin}\\
 E_{sec}^\beta &= E_{in}^\beta - E_{fin}^\beta \label{equation:E_sec}
\end{align}
where $X_{in}$ denotes the initially ionized atom and
$X_{fin1}$ and $X_{fin2}$ describe the two ionized
atoms in the final state. $\beta$ denotes the decay channel characterized
by the quantum numbers of the ionized atoms in the pairs
and triples and $d$ denotes the interatomic distance between the atoms
$X_{fin1}$ and $X_{fin2}$. The initially ionized atom $X_{in}$ can
coincide with one or both of
the final state atoms
$X_{fin1}$ and $X_{fin2}$. As explained in
chapter \ref{chapter:autoionization}, the distribution
of the vacancies over the different
atoms determines the kind of electronic decay process at hand. Hence, in an
Auger process all three atoms would coincide, for an ICD $X_{in}$
would coincide with one of $X_{fin1}$ and $X_{fin2}$ and for an {ETMD}3
all ionized states are located on different atoms.

In all autoionization processes considered, a second electron
is emitted with the kinetic energy $E_{sec}$. If $E_{sec}<0$, then
the final state energy is higher than the initial state energy and the        
process is energetically not accessible. Hence, the corresponding channel     
is closed.
                                                               
This ad hoc approach easily allows to correct for energetic shifts of         
ionization potentials as observed in larger clusters.

\begin{align}
 \Gamma_{ICD}  &= \sum\limits_{i,\beta} N_{ICD,i}  \Gamma_{ICD,i,\beta}\\
 \Gamma_{ETMD} &= \sum\limits_{j,\beta} N_{ETMD,j} \Gamma_{ETMD,j,\beta}\\
\end{align}

$N_{ICD} = N_{in} \cdot N_{fin} = N_{Ar} \cdot N_{Xe}
 = \sum\limits_i N_{ICD,i}$
$N_{ETMD} = N_{in} \cdot N_{fin} (N_{fin} - 1) = N_{Ar} \cdot N_{Xe} (N_{Xe} - 1)
 = \sum\limits_j N_{ETMD,j}$

$N_{ICD,i}$ and $N_{ETMD,j}$ depend on the structure

\begin{equation}
 \Gamma_{\beta}(E_{res}) = 2\pi \left|
                           \braket{\Phi_{in}| H_f |\chi_{\beta}}
                           \right|^2
\end{equation}

\begin{equation}
 \Gamma_{ICD,i,\beta} = 2\pi
                        \frac{\sigma^{(X_E)}(\omega_{vp,\beta})}
                        {R_i^6 \, \omega_{vp,\beta}^4 \tau_{in,\beta}}
\end{equation}


\begin{equation}
 \Gamma_{ETMD,j,\beta} = 2\pi \sum\limits_{m,M_{in}',D}
                        \frac{a_m \Theta_m(\alpha_i) \sigma^{(X_E)}(\omega_{vp,\beta})
                              \tilde{D}_{m,j,\beta}(M_{in,D},M_{in,D'})}
                         {R_j^6 \omega_{vp,\beta}}
\end{equation}


\begin{equation}
\end{equation}


\begin{equation}
\end{equation}


\begin{equation}
\end{equation}


\begin{equation}
\end{equation}


\begin{equation}
\end{equation}



\begin{equation}
\end{equation}


