\section{Discussion
\label{sec:discussion}}
%
%We have discussed the experimental spectra in Section \ref{sec:exp_results} and
%the expected features of spectra for different cluster types in
%Section \ref{sec:th_results}. In this section we combine the experimental and theoretical
%results in order to determine the structure of the noble gas clusters.
%
%The simulated secondary electron spectra for the xenon core structures
%(see Figure \ref{figure:xe_3_in})
%show an ICD peak at higher energies than those structures with added xenon
%atoms on the surface. This does not fit to the experimentally determined
%spectra and hence we consider an argon core to be more probable for the
%small measured clusters.
%
%The experimental spectra in Figure \ref{}
%show a broadening of the secondary electron peak with increasing
%pressure (?) and hence cluster size just as can be expected from
%the theoretically simulated spectra in Figure \ref{figure:surf}.
%
%From the theoretical simulations of larger small clusters
%having the same xenon content as experimentally observed a clearly
%visible ETMD peak is to be expected as shown in Figure \ref{figure:ar_3_6in}.
%However, the experimental spectra do not show a signal in the energy
%region of \unit[1.5--6.5]{eV}. From this we conclude that the clusters
%measured were smaller than 55 atoms containing only very few xenon atoms.
%Additionally, if the cluster contains more than one xenon atom, they are
%not expected to be close because of an otherwise visible ETMD peak
%(compare Figure \ref{figure:2tops}).
%
%The large clusters with a large xenon content can be expected to have
%a xenon core being surrounded by argon layers and hence, both ICD and
%ETMD should be visible in the spectrum. Unfortunately, the ICD and ETMD
%peaks overlap such that the two processes can not be distiguished.
%
We now would like to interpret our experimental findings in view of the simulated results described in Section \ref{sec:th_results}.
Figure \ref{figure:eff} shows that for Ar 3s$^{-1}$ states, autoionization by either ICD or ETMD is the only mode of relaxation except for the clusters with the smallest Xe content.
