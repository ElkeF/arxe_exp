\documentclass[DIN,pagenumber=false,parskip=half,fromalign=left,fromphone=true,fromemail=true,fromurl=false,fromlogo=false,fromrule=false]{scrlttr2}
\usepackage[utf8]{inputenc}
\usepackage{ngerman}
\usepackage[english]{babel}
\usepackage{units}
\usepackage{tabularx}
\usepackage{booktabs}
\usepackage{multirow}
\usepackage{xcolor}
\usepackage{graphicx}
\usepackage[right]{eurosym}
\usepackage{amsmath,amsfonts,amssymb}
\usepackage{braket}
\usepackage{graphicx}
%\RequirePackage{graphicx}

\setkomavar{fromname}{Dr. Elke Faßhauer}
\setkomavar{fromaddress}{Department of Chemistry\\University of Troms\o --- The Artic University of Norway\\9037 Tromsø\\Norway}
\setkomavar{fromphone}{+47 40382485}
\setkomavar{fromemail}{elke.fasshauer@uit.no}
\setkomavar{subject}{Revision of Out Article jp-2016-06665e}
\setkomavar{signature}{Dr. Elke Faßhauer (corresponding author)}


\begin{document}
\begin{letter}{To the Editor}
	
	\opening{Dear Editor,}


Our paper has been reviewed by two of your Referees,
who recommended it for the publication in the
New Journal of Physics subject to minor amendments.
We are thankful for the comments made by the Referees and
respond to them below. Additionally, we took the opportunity to
{\color{blue}{correct some typos or whatever.}}

\textbf{Referee \#1} raised the following points:

\begin{enumerate}
 \item \emph{In general, I find the manuscript clearly written but unnecessary long. Some details and discussions can be avoided without weakening the presentation. Here are some examples:}
  \begin{enumerate}
   \item \emph{ICD is explained twice in the introduction (top of page 3 and
         3rd paragraph of page 3)}\\
   \item \emph{the sentence: The notion of ICD requires that the electronic orbitals … can be distinguished... make a distinction between ICD and Auger decay. does not bring important informations. Furthermore, using only a one-particle picture to define ICD may be misleading. For example, in molecular clusters ICD can be clearly identified after deep ionization (typically for binding energy above 20 eV) of one molecule. However, owing to the strong electronic correlation in this binding energy range the states cannot be described as ionization from a single orbital.}\\
   \item \emph{Page 9: the sentence “These atoms do not necessarily need to form bonds...” just rephrases what is already explained earlier in the manuscript on ICD.}
   \item \emph{Page 10: the definition of open and close channel is repeated. Explanations in page 9 and figure 1 are enough.}
   \item \emph{Page :
         the section “Cluster Structures” can be moved to Supporting information since only Table 2 is important to understand the results. Furthermore, it seems to me that the paragraph on previous study of the authors on large clusters does provide relevant informations on the reported results.}
   \item \emph{Page:
         the section Outer valence spectra is very detailed and the reader may lose the main point of this section which is, as far as I understood, to evaluate experimentally the percentage of Xe in the clusters. All extra details can be avoided. Just to help the readers, this section could be renamed to highlight the goal of the section like for example “Xe content”.}
   \item \emph{Page:
         I think the section Inner valence spectra can be skipped without losing any relevant informations.}
  \end{enumerate}

  \item \emph{Page 8:
        the authors write that “the intensities of the peaks are proportional to the probability decay as well as the decay width”. However, the probability and the width are simply related : Prob =  Width*time/hbar.}\\
        We included the relation between the decay probability and the decay
        width. However, the important information of this sentence is that
        the decay widths can be used to describe the intensities of the
        different peaks. Therefore, its proportionality is important. The
        sentence now reads:\\
        The latter is proportional to the
probability of the decay {\color{blue}{$P=\frac{\Gamma t}{\hbar}$}}
as well as the theoretically determinable
decay width $\Gamma=\frac{\hbar}{\tau}$,
which is inversely proportional to the lifetime $\tau$ of the decay process.

  \item \emph{Page 11:
        Eqs 6 and 7 the sums run over ALL pairs and ALL triples. In that case there should not be $N_{ICD,i}$ and $N_{ETMD,j}$ in the sums; otherwise those pairs and triples are counted several times.}\\
        It now reads:\\
        These are given by the sum over the decay widths of all 
        {\color{blue}{geometrically different}} pairs $i$ for the
        ICD and all {\color{blue}{geometrically different}} triples $j$
        for the ETMD(3),
        respectively, and over all channels $\beta$:

  \item \emph{Page 12:
        some symbols are not defined: $\Phi_{in}$, $\chi_\beta$, $H_f$
        for example.
        Since this part of theory has already been published, it can be shorten
        leaving just Eqs. 9 and 10.
}

  \item \emph{Page 12:
        ionization cross-sections, radiative lifetimes,... are not EXPERIMENTAL properties of atoms, they are intrinsic properties of atoms that can be experimentally measured.}
 
  \item \emph{As far as I understand Figs 4-8 report Widths with respect to Electron energy and are thus not strictly speaking electron emission spectra. The authors should really compute the spectra. Furthermore, the lines which I guess are some convolution of the bars are not defined. And more importantly, how these “spectra” were normalized is not clear: in the cluster with 55 Ar atoms the central and inner layer of atoms can undergo ICD. A more intense band compared to 13 atoms is therefore expected. More explanations are needed.}

  \item \emph{Page 18:
        it is stated that the decay width decreases like R-6 but in page 20 it decreases exponentially.
}

  \item \emph{Page 29:
        the efficiency of ICD is found below unity for clusters having few Xe atoms. I guess this is because in these clusters ICD and ETMD are closed for some Argon atoms which are anyway ionized. Maybe a statistical model may explain it. The theory could also help since it should be easy to determine the efficiency for all clusters considered in this study.}


\end{enumerate}

\textbf{Referee \#2} raised the following points:

\begin{enumerate}
 \item \emph{ETMD(3) should be explained in the introduction by a sentence or two to aid the reader (at the moment, it is first explained in the theory section).
}
 \item \emph{The authors mention that the ‘nuclear dynamics may play an important role’ but that they have been neglected in the calculations. While I understand that it is not feasible in such large systems the authors should at least briefly discuss the possible effect of nuclear dynamics e.g. peak shifts, broadening, ICD vs. ETMD, etc. }

 \item \emph{The spin-orbit splitting of the Ar 3p$^{-1}$ state is briefly mentioned in Table 3 in the experimental section but it also stated that the a single value of the binding energy of Ar 3p equal to 15.3 eV is used for all of the simulations. The authors should explain the motivation for this choice in the theory section.}

 \item \emph{Page 13:
       The authors should justify the use of the ‘unified force field’ method for the calculation f the cluster structure (p13).}

 \item \emph{Page 12:
       On page 12 last line there is to Tables II and III. I have not been able to locate these. The reference should be clarified.}

 \item \emph{Page 20,21:
       it is not clear what structure of the larger clusters with 3871 atoms is used in the simulation. The authors mention icosahedral and cuboctahedral but no further details are given, e.g. Ar-Xe distance, Ar-Ar distance, etc.}

 \item \emph{In fig. 9(a) it is specified in the text that  Ar <N> =190 for the black trace but as far as I can tell no value is given for the other spectrum on the same graph.}

 \item \emph{It is not clear what the dotted lines in Figure 10 are?}

 \item \emph{The colour schemes of some of the figures lead to confusion when printed in grayscale. Some effort could be made to resolve this.}

\end{enumerate}


I took the opportunity to shorten the title in order to not highlight only
one out of several aspects. It now reads:\\
      Non-nearest Neighbour ICD in 
      Clusters


        \closing{Sincerely yours,}
	\end{letter}

\end{document}
