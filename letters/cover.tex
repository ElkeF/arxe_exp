\documentclass[DIN,10pt,pagenumber=false,parskip=half,fromalign=left,fromphone=false,fromemail=true,fromurl=false,fromlogo=true,fromrule=false]{scrlttr2}
\usepackage[utf8]{inputenc}
\usepackage{ngerman}
\usepackage{units}
\usepackage{tabularx}
\usepackage[right]{eurosym}
\usepackage{graphicx}
\usepackage[top=2cm,bottom=0cm,left=2cm,right=2cm]{geometry}
%\RequirePackage{graphicx}

\setkomavar{fromname}{Dr. Elke Faßhauer}
\setkomavar{fromaddress}{CTCC\\ University of Tromsø{}
                         --- The Arctic University of Norway
                         \\ 9037 Tromsø \\ Norway}
%\setkomavar{fromphone}{06221/545220}
\setkomavar{fromemail}{elke.fasshauer@uit.no}
\setkomavar{subject}{Manuscript Submission for Journal of Physical Chemistry C}
\setkomavar{signature}{Elke Faßhauer}
\setkomavar{fromlogo}{\includegraphics[width=3cm]{uit.pdf}}


\begin{document}

\begin{letter}{Journal of Physical Chemistry\\Editorial Board}
\textheight200mm
\opening{Dear Editor,}

I hereby submit our manuscript entitled
\emph{Long-range Interatomic Coulombic Decay in ArXe clusters: Experiment and
Theory} written by M. Förstel, M. Mucke, T. Arion, T. Lischke, M. Pernpointner,
Uwe Hergenhahn and E. Fasshauer for consideration of publication in
the Journal of Physical Chemistry C on behalf of all authors.

In this joint theoretical and experimental work we study two competitive
electronic decay processes in clusters for which we have chosen ArXe clusters
as a model system. We find that the Electron Transfer Mediated Decay (ETMD(3)),
which is a pure interface effect,
is dominated by the long-range Interatomic Coulombic Decay (ICD)
decay mechanism. Due to these results we
are able to draw the conclusion, that the cluster structure provides a clear
interface between argon and xenon atoms instead of mixed cluster structures.

\closing{Yours sincerely,}
\end{letter}

\end{document}
