\section{Experimental}
%
\begin{table}
\caption{
Expansion parameters used for cluster production. Here, Xe$_{in}$ is the fraction (particle number) of Xe in the expanding gas before the expansion, $T$ the nozzle temperature, and $p$ the stagnation pressure. Experiments were done with $d = 80~\mu$m (first two sections) and $d = 100~\mu$m (bottom section) conical nozzles of 15$^\circ$ half opening angle. For the mixed clusters, $\langle N_{Ar} \rangle$ and $\langle N_{Xe} \rangle$ refer to hypothetical cluster sizes yielded with an expansion of the respective pure gas at the conditions given. These values are calculated as in ref.\ \protect\citenum{hagena1981}. Due to the much lower freezing point of Ar, we basically have an Ar seeded expansion of Xe gas. We therefore expect actual cluster sizes in-between the two limiting values given. Inaccuracies in the calculation of $\langle N\rangle$ due to fluctuations of the input parameters are less than 6\,\%. This figure does not include systematic errors of the empirical model used.
}
\label{tab:cluster}

\begin{tabular}{l c c c r r}
%
\toprule
  label  &  Xe$_{in}$ [\%]  &  $T$ [K]  &  $p$ [bar] & $\langle N_{Ar} \rangle$ & $\langle N_{Xe} \rangle$ \\
%
\midrule
% Ar, from Marko 
Ar (1)  & --- &  96.5  & 0.35  &  42  &  --- \\
% Ar, from Marko 
Ar (2)  & --- &  96.5  & 0.67  & 190  &  --- \\
% 1103 676, 679
 (a) & 1.2 &  174   & 0.32  &   4  &   62 \\
% 1103 670
 (b) & 1.2 &  174   & 0.49  &  10  &  168 \\
% 1103 671
 (c) & 1.2 &  174   & 0.68  &  21  &  362 \\
% 1103 663
 (d) & 3.0 &  172   & 0.28  &   3  &   48 \\
% 1103 640
 (e) & 5.0 &  171   & 0.51  &  12  &  202 \\
% 0506 382, Keller offset ber?cksichtigt
Xe (1)  & 100 & 183.5  & 0.68  & ---  &  517 \\     
\midrule
% 1103 653
1103, only coinc  & 3.0 &  172   & 0.51  &  11  &  196 \\
% 1103 658
  & 3.0 &  172   & 0.68  &  22  &  385 \\
% 1103 630
  & 5.0 &  167   & 0.37  &   6  &  108 \\
% 1103 629
  & 5.0 &  167   & 0.68  &  26  &  451 \\
\midrule
% 1004 867
1004, 867  & 2.5 &  154   & 2.12  &  920 & 15760\\
% 1004 868
868  & 2.5 &  158   & 1.50  &  355 &  6090\\
% 1004 886
886  & 2.5 &  161   & 2.41  &  980 & 16770\\
% 1004 858
858  & 5.0 &  149   & 2.50  & 1620 & 27700\\
%
\bottomrule
\end{tabular}
\end{table}
%
%
The apparatus used for the experiments consists of a cooled, supersonic molecular jet apparatus and a magnetic bottle spectrometer, which detects photoelectrons and secondary electrons produced after ionization with synchrotron radiation. A detailed description can be found in Ref. \citenum{arion}, and here we focus on those details specific for the current experiment. Commercial Ar and Xe gas was used. Separate containers for the two gases were filled up to pressures suitable for producing a certain mixing ratio. The gases were then allowed to mix before the expansion. Expansion parameters and mean cluster sizes according to an empirical model are given in Tab.\ \ref{tab:cluster}. We note that currently no such empirical model for heterogeneous expansions exists. We therefore can only give the mean size for a pure jet of one or the other gas at the given conditions. 

The expansion chamber for the supersonic is separated from the interaction chamber by a non-magnetic, conical skimmer with a diameter 1~mm opening (Beam Dynamics). At few cm distance behind the skimmer, the cluster jet was crossed by synchrotron radiation from the BESSY electron storage ring at Helmholtz-Zentrum Berlin. Electrons were detected by a short 'magnetic bottle� time-of-flight spectrometer, that has been described earlier.\cite{mucke_review} Experiments were carried out at the TGM-4 (last section in Tab.\ \ref{tab:cluster}) and the UE112-PGM-1 (first two sections) beamlines with linearly, horizontally polarized radiation.

