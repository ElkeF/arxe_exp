\section{Experimental}
%
\begin{table}
\caption{
Expansion parameters used for cluster production. Here, Xe$_{\rm in}$ is the molar fraction of Xe in the gas mixture before the expansion, $T$ is the nozzle temperature, and $p$ the stagnation pressure. Experiments were done with $d = 80~\mu$m (first two sections) and $d = 100~\mu$m (bottom section) conical nozzles of 15$^\circ$ half opening angle. For the mixed clusters, $\langle N_{\rm Ar} \rangle$ and $\langle N_{\rm Xe} \rangle$ refer to cluster sizes arrived at by an expansion of the respective pure gases at the given conditions. These values are calculated as in ref.\ \protect\citenum{hagena1981}. Due to the much lower freezing point of Ar, we basically have an Ar seeded expansion of Xe gas. We therefore expect actual cluster sizes in-between the two limiting values given. Inaccuracies in the calculation of $\langle N\rangle$ due to fluctuations of the input parameters are less than 6\,\%. This figure does not include systematic errors of the empirical model used.
}
\label{tab:cluster}

\begin{tabular}{l c c c c r r}
%
\toprule
  \multicolumn{2}{r}{size label}  &  Xe$_{\rm in}$ (\%)  &  $T$ (K)  &  $p$ (bar) & $\langle N_{\rm Ar} \rangle$ & $\langle N_{\rm Xe} \rangle$ \\
%
\midrule
% Ar, from Marko 
Ar & (1) & --- &  96.5  & 0.35  &  42  &  --- \\
% Ar, from Marko 
Ar & (2) & --- &  96.5  & 0.67  & 190  &  --- \\
% 1103 676, 679
ArXe & S & 1.2 &  174   & 0.32  &   4  &   62 \\
% 1103 670
ArXe & M & 1.2 &  174   & 0.49  &  10  &  168 \\
% 1103 671
ArXe & L & 1.2 &  174   & 0.68  &  21  &  362 \\
% 1103 663
ArXe & S & 3.0 &  172   & 0.28  &   3  &   48 \\
% 1103 640
ArXe & M & 5.0 &  171   & 0.51  &  12  &  202 \\
% 0506 382, Keller offset ber?cksichtigt
Xe &  & 100 & 183.5  & 0.68  & ---  &  517 \\     
\midrule
% 1103 653
ArXe & M & 3.0 &  172   & 0.51  &  11  &  196 \\
% 1103 658
ArXe & L & 3.0 &  172   & 0.68  &  22  &  385 \\
% 1103 630
ArXe & S  & 5.0 &  167   & 0.37  &   6  &  108 \\
% 1103 629
ArXe & L  & 5.0 &  167   & 0.68  &  26  &  451 \\
\midrule
% 1004 867 (32 eV)
ArXe & XL & 2.5 &  154   & 2.12  &  920 & 15760\\
% 1004 868
%868 & & 2.5 &  158   & 1.50  &  355 &  6090\\
% 1004 886 (17 eV)
%886 &  & 2.5 &  161   & 2.41  &  980 & 16770\\
% 1004 858
%858 &  & 5.0 &  149   & 2.50  & 1620 & 27700\\
%
\bottomrule
\end{tabular}
\end{table}
%
%
The apparatus used for the experiments consists of a supersonic molecular jet with a cooled nozzle, and a magnetic bottle spectrometer, which detects photoelectrons and secondary electrons produced after ionization with synchrotron radiation.\cite{arion} 
% Leave this citation in place to avoid Latex error
A detailed description can be found in Ref. \citenum{arion}, and here we focus on details specific for the current experiment. Commercial Ar and Xe gas was used. 
Separate containers for the two gases were filled up to pressures suitable for producing a certain mixing ratio. The gases were then allowed to mix before the expansion. 
Expansion parameters and mean cluster sizes are given in Table \ref{tab:cluster}. 
Currently, no model to predict cluster sizes in a heterogeneous gas expansion exists. 
We therefore can only give the mean size for a pure jet of one or the other gas at the given conditions, which we have derived according to an empirical scaling law.\cite{hagena1981} 
We note that recent analyses of photoionization spectra suggests that  mean sizes arrived at by scaling laws are smaller than the actual size, at least for rare gas clusters.\cite{bergersen,hergenhahnprb,foerstel_arg2_2011}

The expansion chamber for the supersonic jet is separated from the interaction chamber by a non-magnetic, conical skimmer with a diameter 1~mm opening (Beam Dynamics). 
At few cm distance behind the skimmer, the cluster jet was crossed by synchrotron radiation from the BESSY electron storage ring at Helmholtz-Zentrum Berlin. 
Electrons were detected by a short `magnetic bottle' time-of-flight spectrometer, that has been described earlier.\cite{mucke_review} 
Data were recorded in two different beamtimes at the UE112-PGM-1 (small and medium sized clusters, first two sections in Table \ref{tab:cluster}) and at the TGM-4 beamline (large clusters, last section in Table \ref{tab:cluster}). 
Linearly, horizontally polarized radiation was used. 
The storage ring was operated in single bunch conditions.
